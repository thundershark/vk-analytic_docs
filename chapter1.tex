
\begin{chapter1}
\newpage
\section{Введение}
В настоящий момент довольно остро стоит вопрос о сохранении тайны связи при использовании электронной почты, чата, социальных сетей и иных электронных средств коммуникаций. В настоящий момент  закон о сохранении тайны связи не охватывает публичные сервисы.\footnote{ Тайна связи, электронная почта и российские суды (http://www.securitylab.ru/blog/personal/emeliyannikov/37733.php)}
Кроме того, опубликованные Эдвардом Сноуденом данные наглядно демонстрируют, что межправительственные системы слежения (созданные для борьбы с терроризмом) используются для достижения экономических и политических целей, и нарушают права граждан на тайну частной жизни и тайну переписки.\\

Целью проекта является создание веб-приложения, демонстрирующего различные возможности по сбору сведений об отдельном человеке с использованием только открытых источников.\footnote{Правительство США предало интернет. Нам надо вернуть его в свои руки     (http://habrahabr.ru/post/192852/)}\footnote{Эдвард Сноуден (http://ru.wikipedia.org/wiki/Сноуден,\_Эдвард)} 
Особенно интересным представляется создать веб-сервис для автоматизированного анализа страницы в социальной сети с выделением дополнительных сведений о человеке, на основе сведений о его друзьях. Веб-сервис планируется создать в демонстрационный целях т.к. решение о публикации своих данных осуществляется непосредственно человеком.\\

Люди часто недооценивают значение метаданных и комплексного анализа отдельных фактов. Они не задумываются о том, что вступая в определенные электронные сообщества их личные данные может сообщить не только владелец аккаунта, но и другие участники сообщества. Причем, чем ближе они знакомы, тем больше данных они могут передать, иногда даже сами того не подозревая.\\

Данный сервис задуман с целью проверки оценки уровня защищенности персональной информации, которую пользователь оставляет конфиденциальной становясь участником виртуального сообщества, но которая может быть получена в результате анализа косвенных источников. \\

Данный сервис не является социально опасным по следующим причинам:
\begin{itemize}
\item пользователь сервиса имеет возможность анализа только той страницы, для которой известны данные авторизации;
\item сервис безопасен для пользователя т. к. авторизация происходит по средствам  API  социальной сети и данные авторизации не передаются на сервер приложения; 
\item мировой опыт показывает, что уже созданы куда более мощные средства для анализа данных. Однако, все они являются достоянием специальных служб. Данный сервис является попыткой защитить конечного пользователя, демонстрируя ему часть той информации, которую о нем могут собрать соответствующие службы.
\end{itemize}

\section{Основной функционал приложения}
Обязательный функционал позволит определить пол, возраст, ВУЗ некоторого человека в социальный сети Вконтакте, на основе данных получаемых в автоматическом режиме. Состав дополнительного функционала, сообщающий значимую дополнительную информацию о человеке,  будет определен в процессе разработки, т.к. на начальном этапе не представляется возможным определить его из-за большого размера проекта социальной сети Вконтакте.

Оценка уровня конфиденциальности закрытых персональных данных пользователя на основе активности в социальной сети

\subsection{Цели и задачи дипломного проекта}
Задачи:
	\begin{itemize}
\item анализ лигитимности функционала приложения;
\item анализ существующих web-сервисов, которые предоставляют дополнительную информацию о пользователе с помощью анализа косвенных признаков;
\item анализ существующих научных подходов для реализация данной задачи;
\item составление описания для каждого решения;
\item анализ законности существования приложений данного типа;
\item анализ существующих научных подходов для реализация данной задачи;
\item реализация обязательного функционала. Уточнние и реализация дополнительного функционала;
\item тестирование и доработка приложения.
	\end{itemize}
	
\section{Анализ существующих решений}
Вследствии огромной популярности социальных сетей, в интернете уже давно стали появляться проекты, дополняющие их функционал. Такие проекты решают иные задачи с использованием инструментария, предоставляемого социальными сетями. Помимо этого, существуют проекты, обобщающие функционал различных социальных сетей в одном месте. Такие проекты получили название агригаторов.\\

Был проведен анализ существующих решений, которые с помощью косвенных данных и методов автоматического анализа позволяют «вычислить» дополнительную информацию о человеке, которую он не указывал в явном виде, найти на web-ресурсах информацию не доступную обычным поисковым системам, получить релевантную информацию которая обычно слишком низко ранжируется.
\subsection{Отдельные web-приложения для поиска людей}
В сети Интернет представлен ряд приложений для поиска аккаунтов людей сразу во множестве социальных сетей. Стоит отметить, что в данный момент количество социальных сетей уже исчисляется десятками и это только те, которые имеют крупное и живое сообщество. Все сети имеют свои особенности,  поэтому агрегация этого многообразия - задача не простая, и ее можно решить несколькими способами.
\begin{itemize}
\item \subsection{http://people.yandex.ru} %add picture
people.yandex.ru – простой сервис (в одну строку) поиска в различных социальных сетях с преобладающим числом  русскоязычных пользователи. Также сервис обладает функционалом кластеризации результатов.  Данный сервис не содержит функционала,  похожего на мое приложения, но является неплохим примером построения интерфейса взаимодействия с пользователем;
\item \subsection{http://topsy.com}
topsy.com - realtime поисковая система, специализирующаяся на поиске по социальным медиа, таким как блоги, твиттер, сообщения в социальных сетях;
\item \subsection{http://qwant.com}
qwant.com — поисковая система с особым подходом к ранжированию и поиском по англоязычным социальным сетям (в этом она напоминает people.yandex.ru);
\item \subsection{http://spokeo.com}
spokeo.com — сайт для поиска людей, аггрегирующий информацию из множества других он-лайн и офф-лайн источников, таких как: телефонные справочники, социальные сети, фотоальбомы, маркетинговые исследования, списки рассылки, государственные переписи, безнесс-сайты, всего — более чем из 60 источников. Основные базы для поиска на английском языке и, как следствие, позволяет довольно точно отследить людей, пользующихся иностранными сайтами в повседневной жизни.
\end{itemize}

\subsection{Сервисы анализа сообществ и трендов в социальных сетях}
В интернете содержится огромное количество книг, инструкций и примеров психологических анализов страницы из социальной сети, но сервисы для автоматизации этого процесса практически отсутствуют. Это можно объяснить тем, что на такого рода сервисы сложно манетизировать. Естественно, что у самихвладьцев есть подобные и даже куда мощные средства. Так например система матрикснет от Яндекс умеет классифицировать следующим образом пользователей.\\ %Ссылка на матрикс

Данный класс приложений похож на мое приложение тем, что с помощью автоматических алгоритмов  он анализирует состояние и изменения в сообществах и социумах, в то время как я анализирую отдельного человека. Некоторые из этих приложений уникальны и весьма интересны, и на основании этого включены в анализ. Интересно что много сервисов для анализа twitter`а являются некоммерческими и вследствии этого быстро теряли поддержку, так например в 2011 году эти сервисы еще существовали или были популяярны и хорошо работали:
\begin{itemize}
\item \subsubsection{TweetStats}
TweetStats - показывает количество сообщений по месяцам, частоту сообщений в зависимости от времени дня и дня недели. Проект некоммерческий, не развивается, некоторые функции работают не стабильно;
\item \subsubsection{Twinfluence}
ыл простым инструментом для измерения совокупного влияния твиплов и их фолловеров, а также в качестве бонуса предоставляет статистику некоторых социальных сетей. В данный момент недоступен, по доменну на котором находился проект стоит переадрисацию на компанию в которой работают бывшие владельцы Twinfluence;
\item \subsubsection{TweetEffect}
TweetEffect – отражал изменение количества фоловеров после каждого сообщения. Сервис перестал работать после изменения в twitter API;
\item \subsubsection{Twitteranalyzer}
Twitteranalyzer - статистика по  направлениям: Пользователи, Друзья, Упоминания, Группы и более мелким подуровням, что позволяет получить довольно много информации для анализа; Так же перестал работать
\item \subsubsection{•}
\item \subsubsection{•}
\item \subsubsection{•}\\
\item \subsubsection{•}\\
\item \subsubsection{•}
\item \subsubsection{•}
\item \subsubsection{•}
\end{itemize}


\end{chapter1}
