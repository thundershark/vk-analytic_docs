
\begin{chapter1}
\newpage
\section{Введение}
В настоящий момент довольно остро стоит вопрос о сохранении тайны связи при использовании электронной почты, чата, социальных сетей и иных электронных средств коммуникаций. В настоящий момент  закон о сохранении тайны связи не охватывает публичные сервисы.\footnote{ Тайна связи, электронная почта и российские суды (http://www.securitylab.ru/blog/personal/emeliyannikov/37733.php)}
Кроме того, опубликованные Эдвардом Сноуденом данные наглядно демонстрируют, что межправительственные системы слежения (созданные для борьбы с терроризмом) используются для достижения экономических и политических целей, и нарушают права граждан на тайну частной жизни и тайну переписки.\\

Целью проекта является создание веб-приложения, демонстрирующего различные возможности по сбору сведений об отдельном человеке с использованием только открытых источников.\footnote{Правительство США предало интернет. Нам надо вернуть его в свои руки     (http://habrahabr.ru/post/192852/)}\footnote{Эдвард Сноуден (http://ru.wikipedia.org/wiki/Сноуден,\_Эдвард)} 
Особенно интересным представляется создать веб-сервис для автоматизированного анализа страницы в социальной сети с выделением дополнительных сведений о человеке, на основе сведений о его друзьях. Веб-сервис планируется создать в демонстрационный целях т.к. решение о публикации своих данных осуществляется непосредственно человеком.\\

Люди часто недооценивают значение метаданных и комплексного анализа отдельных фактов. Они не задумываются о том, что вступая в определенные электронные сообщества их личные данные может сообщить не только владелец аккаунта, но и другие участники сообщества. Причем, чем ближе они знакомы, тем больше данных они могут передать, иногда даже сами того не подозревая.\\

Данный сервис задуман с целью проверки оценки уровня защищенности персональной информации, которую пользователь оставляет конфиденциальной становясь участником виртуального сообщества, но которая может быть получена в результате анализа косвенных источников. \\

Данный сервис не является социально опасным по следующим причинам:
\begin{itemize}
\item пользователь сервиса имеет возможность анализа только той страницы, для которой известны данные авторизации;
\item сервис безопасен для пользователя т. к. авторизация происходит по средствам  API  социальной сети и данные авторизации не передаются на сервер приложения; 
\item мировой опыт показывает, что уже созданы куда более мощные средства для анализа данных. Однако, все они являются достоянием специальных служб. Данный сервис является попыткой защитить конечного пользователя, демонстрируя ему часть той информации, которую о нем могут собрать соответствующие службы.
\end{itemize}

\section{Основной функционал приложения}
Обязательный функционал позволит определить пол, возраст, ВУЗ некоторого человека в социальный сети Вконтакте, на основе данных получаемых в автоматическом режиме. Состав дополнительного функционала, сообщающий значимую дополнительную информацию о человеке,  будет определен в процессе разработки, т.к. на начальном этапе не представляется возможным определить его из-за большого размера проекта социальной сети Вконтакте.

Оценка уровня конфиденциальности закрытых персональных данных пользователя на основе активности в социальной сети
\subsection{Цели и задачи дипломного проекта}
Задачи:
\begin{itemize}
\item анализ лигитимности функционала приложения;
\item анализ существующих web-сервисов, которые предоставляют дополнительную информацию о пользователе с помощью анализа косвенных признаков;
\item анализ существующих научных подходов для реализация данной задачи;
\item составление описания для каждого решения;
\item анализ законности существования приложений данного типа;
\item анализ существующих научных подходов для реализация данной задачи;
\item реализация обязательного функционала. Уточнние и реализация дополнительного функционала;
\item тестирование и доработка приложения.
\end{itemize}
\end{chapter1}
